\documentclass[]{letter}
\usepackage{amssymb}
\usepackage{amsmath}
\usepackage{amsfonts}
\usepackage{wasysym}
\usepackage[latin1]{inputenc}

\begin{document}


\begin{letter}
	
In August 2010, the Wide-Field Infrared Survey Explorer (WISE) spacecraft completed its primary mission of surveying the entire sky at four mid-infrared wavelength range bands: 3.4, 4.6, 12 and 22 $\mu$m. These are called the \emph{W1, W2, W3} and \emph{W4} bands respectively. Since WISE detects light emitted in the infrared, it has been used to produce very good maps of the interstellar dust within the Milky Way Galaxy. Each of the \textit{W1,\dots,W4} filters detects emission from different sources; the shortest two wavelengths are emitted mostly by stars, while the longer wavelengths are from dust clouds.
\\
\\
The \textbf{central bulge} of the Galaxy is an oblate spheroid central structure approximately 0.9~kpc thick and with an axis ratio~$b/a \sim0.61$, with its long axis lying in the plane of the disk of the Galaxy. The mass estimate of the central bulge is within a range of about 12.5 to 16 billion solar masses. Almost half of the stars in the galactic bulge are on orbits that give the bulge an irrefutable X-shape, a feature seen most prominently in the W1 and W2 bands of the WISE dataset. The boxy/peanut shape of the Milky Way's core was first revealed by the Diffuse Infrared Background Experiment, onboard the COBE satellite.
The X-shape of the bulge reinforces the idea that the Galaxy has suffered no major mergers or collisions with other galaxies within the past ten billion years; any major collision or merger event would have disrupted the structure we currently see.	
This structure has its two arms crossing about 500 pc away from the center, and, moreover, is identifiable in over 50\% of external galaxies.
\\
\\
Overall, the Milky Way Galaxy is a highly luminous, reddish barred spiral galaxy. This high luminosity compared to other galaxies of the same morphology is consistent with its high circular velocity of $\sim240$ km.s$^{-1}$. Despite having this high circular velocity, the Milky Way Galaxy is kinematically `cold', in the sense that the velocity dispersions of stars near the Sun are $\sim 25\,\text{km.s}^{-1}$, which is much lower than the rotation speed of the Galaxy.

The Milky Way has been shown to be a barred spiral galaxy, with a central box/peanut shaped bulge, or X-shaped bulge as indicated by a bimodal distribution of red clump giant stars. As viewed from Earth, the left-hand side of the box (on the side of positive galactic longitude) is noticeably larger than the right, due to the perspective effect of the central bulge being tilted towards Earth by approximately 90\textdegree.
\\
\\

In my previous research, NEOWISE images were used to identify and highlight the boxy/peanut morphology of the center of the Milky Way Galaxy. The images were processed in similar ways to previous research using spatial mapping, and my research confirmed these findings quantitatively, by showing that the central bulge is significantly brighter at positive longitudes, than at negative. 

\end{letter}
\end{document}
